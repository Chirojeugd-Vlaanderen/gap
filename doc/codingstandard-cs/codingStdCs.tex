\documentclass[a4paper,11pt]{article}
\usepackage[dutch]{babel}
\usepackage[pdftex]{graphicx}
\usepackage{multirow}
\usepackage[absolute]{textpos}
\usepackage{enumitem}
\usepackage{listings}
\lstloadlanguages{[Sharp]C}

\newcommand{\opmerking}[1]{\marginpar{\scriptsize #1}}

\setlength{\TPHorizModule}{25mm}
\setlength{\TPVertModule}{\TPHorizModule}
\textblockorigin{15mm}{15mm}

% Sans-serif font, volgens huisstijl
%
\renewcommand{\familydefault}{\sfdefault}

\begin{document}

%% algemene instellingen voor listings
\lstset{language=[Sharp]C, tabsize=3, frame=single, basicstyle=\small, texcl=true}

%%
%% Eerst een blok rommel voor de 'huisstijlheader'
%%

\begin{textblock}{2}(0,0)
\includegraphics[height=32mm]{chiro-grijs.jpg} \\
\end{textblock}

\begin{textblock}{4.1}(3,0)
\textbf
{
\begin{tabular}{|p{65mm}|p{10mm}|r|}
\multicolumn{3}{r}{\Large Chirojeugd-Vlaanderen vzw} \\
\multicolumn{3}{r}{\,} \\
\hline
\multicolumn{2}{|l|}{Werkgroep Informatica} & 271 \\
\hline
25 april 2008 & Doc. & 2 \\
\hline
\end{tabular} 
}
\end{textblock}

\vspace*{4mm}

%%
%% HIER BEGINT HET DOCUMENT
%%


\noindent {\Large \bf CODING STANDARD C\#, doc. v. 0.2}

\tableofcontents

\section{Over dit document}

Dit document geeft een aantal richtlijnen voor de ontwikkeling van C\#-applicaties, zowel over
opmaak en naamgeving als over de manier van dingen implementeren.

Het document is opgesteld naar aanleiding van de ontwikkeling van een nieuw Chirogroepprogramma.
De bedoeling is dat de richtlijnen in dit document (en zijn opvolgers) toegepast zullen worden
in alle C\#-projecten die nog voor de Chiro ontwikkeld zullen worden.

\section{Architectuur}

\begin{enumerate}
\item We gebruiken een lagenmodel; iedere laag staat op zich, en communiceert slechts met de
andere lagen via publieke klasses en interfaces.\opmerking{Of enkel interfaces?}
\begin{itemize}
\item Data Access Layer
\item Business Layer
\item User Interface
\end{itemize}
\item Groepeer gerelateerde klasses in class libraries.
\item Gebruik namespaces met een zinvolle naam.
\end{enumerate}

\section{`Opmaak' van cs-bestanden}

\subsection{Gegenereerde code}

\begin{enumerate}[resume]
\item Wijzig geen automatisch gegenereerde code.  Gebruik partial classes als je automatisch
gegenereerde code moet aanpassen.
\end{enumerate}

\subsection{Opbouw}

Zie listing \ref{lblLstOpbouw} voor voorbeelden.

\begin{enumerate}[resume]
\item Elke source file begint met een blokje commentaar met daarin\opmerking{Dit hebben we niet
afgesproken, maar het lijkt me een zinvol idee.}
\begin{itemize}
\item een korte omschrijving van wat er in de code staat
\item een `copyright notice'
\item een verwijzing naar de licentievoorwaarden (?)
\item contactinfo van de verantwoordelijke voor het bestand
\end{itemize}
\item Groepeer de imports van namespaces bovenaan
\item Hoogstens \'e\'en klasse en hoogstens \'e\'en namespace in een source file.
\item Gebruik \lstinline !#region! \ldots \lstinline !#endregion!-constructies om stukken code
die bij elkaar horen te groeperen.
\item Vermijd files van meer dan 500 lijnen code.\footnote{Uitzondering is automatisch gegenereerde
code}
\item Vermijd lijnen die langer zijn dan 120 karakters.
\end{enumerate}

\begin{lstlisting}[float, caption=Opbouw, texcl=false, label=lblLstOpbouw]
// Groep.cs - basisklasse voor ChiroGroep en Satelliet
// Copyright (C) 2008, Chirojeugd Vlaanderen
//
// (licentievoorwaarden iemand?)
//
// Contact: Johan Vervloet <johan.vervloet@chiro.be>
//

import System.Data;
import System.Data.Linq;
import System.Data.Linq.Mapping;
import Kipadmin.StamNrTools; 

namespace ChiroData
{
	public class Groep
	{
#region Constructors
		// definieer hier de constructors
#endregion

		// nog meer definities...
	}
}
\end{lstlisting}

\subsection{Whitespace en indentatie}

Voorbeelden in listing \ref{lblLstWhitespace}.

\begin{enumerate}[resume]
\item Geen spaties na een (rond) haakje-open of voor een (rond) haakje-toe
\item Scheid blokken gerelateerde code van elkaar door \'e\'en lege lijn.
\item Gebruik tabs voor indentatie
\item Indenteer ook commentaar
\item Voor `blokken' code tussen accolades, zijn de accolades ge\"indenteerd op het niveau van
de bovenliggende structuur.  Op een regel met een accolade staat enkel die accolade.  Wat binnen de accolades staat, is ge\"indenteerd.
\item Een spatie tussen \lstinline !if!, \lstinline !for!, \lstinline !while! en het haakje-open
daarna.  Bijv: \lstinline !if (voorwaarde)!
\item Geen spatie tussen de naam van een method en het haakje daarna.  Wel een spatie na de komma's tussen verschillende argumenten.  

Bijv.: \lstinline
!DoeIets(argument, tweedeArgument)!.
\end{enumerate}

\begin{lstlisting}[float, caption=Indentatie, label=lblLstWhitespace]
public static void Main(string[] argumenten)
{
	Console.WriteLine("Hallo wereld!");

	if (argumenten.Count() == 0)
	{
		// Commentaar is mee ge\"indenteerd.
		// Let op de positie van de accolades!

		Console.WriteLine("Geen argumenten.");
	}
}	
\end{lstlisting}

\subsection{Controlestructuren en method calls}

\begin{enumerate}[resume]
\item Gebruik steeds accolades bij een \lstinline !if!, \lstinline !while!, \lstinline
!for!,\ldots, ook als er maar 1 (of geen) statement moet staan.  Zie listing
\ref{lblLstControle}.
\begin{lstlisting}[float, caption=Controlestructuren, label=lblLstControle]
// FOUT want verwarrend:
for (i = 0; i < N; a[i++] = 0);
while (eenOfAnder());

// GOED:
for (i = 0; i < N; i++)
{
	a[i] = 0;
}
while (eenOfAnder())
{
	// doe niets
}
\end{lstlisting}
\item Voorzie steeds een \lstinline !default! case bij een \lstinline !switch!-constructie.  Zie
listing \ref{lblLstDefault}
\begin{lstlisting}[float, caption=Steeds een `default', label=lblLstDefault]
switch (isnschrijver.Soort)
{
	InschrijvingsType.Deelnemer:
	{
		// doe het een
	}
	InschrijvingsType.Begeleider:
	{
		// doe het ander
	}
	default:
	{
		// handel onverwachte situatie af
	}
}
\end{lstlisting}
\item Voorzie steeds een \lstinline !else! zonder voorwaarde in een \lstinline !if! \ldots
\lstinline !else if! \ldots constructie.
\end{enumerate}

\subsection{Commentaar}

\begin{enumerate}[resume]
\item Plaats vlak voor elke method een blok xml-commentaar die de werking van de method
omschrijft.\footnote{Visual Studio maakt automatisch een template aan als je op de lijn voor de
methoddeclaratie \lstinline !///! intikt.}
\item Zet in methods enkel commentaar waar het nodig is, om de werking van je code te
verduidelijken.  Als je leesbare code schrijft, kan je de hoeveelheid commentaar beperken.
Complexe code moet goed gedocumenteerd zijn.
\item Gebruik \lstinline !//! of \lstinline !///! (voor XML-commentaar) voor `structureel'
commentaar in je code.  \lstinline !/*! en \lstinline !*/! kan je dan nog gebruiken om tijdelijk
een stuk code uit te commentari\"eren.
\item Schrijf commentaar in duidelijk en correct Nederlands :)
\item Gebruik 'TODO:' om aan te geven dat er nog iets moet gebeuren.\footnote{De typeische
dingen die achterwege blijven als de deadline eigenlijk voorbij is ;)}
\end{enumerate}

\section{Naamgeving}

\subsection{Namen kiezen}

\begin{enumerate}[resume]
\item Alle eigen gekozen namen zijn in het Nederlands. 
\item Indien een naam uit meerdere woorden bestaat, worden de woorden aan elkaar geschreven,
waarbij elk nieuw woord met een hoofdletter begint.  Ook in samenstellingen start elk
`deelwoord' met een hoofdletter.\footnote{Deze laatste regel is niet absoluut; zo kan bijv. in
\lstinline !leeftijd! de `t' een kleine letter blijven, hoewel dit een samenstelling is.  Het is moeilijk om hier een regel op te
plakken.}
\begin{itemize}
\item \lstinline !geboorteDatum!
\item \lstinline !groepsLeiding!
\end{itemize}
\item Geen hoofdletters in het midden van letterwoorden of afkortingen
\begin{itemize}
\item \lstinline !isVzw!
\item \lstinline !adNr!
\end{itemize}
\end{enumerate}

\subsection{Klasses algemeen}

\begin{enumerate}[resume]
\item Namen van klasses zijn in het enkelvoud, en beginnen met een hoofdletter.
\item Namen van member variabelen beginnen met een kleine letter.\opmerking{Mogelijk
voegen we hier aan toe dat ze het prefix 'm' of '\_' moeten krijgen, om ze te kunnen
onderscheiden van lokale variabelen.}
\item Als een klasse een identificatienummer heeft, dan krijgt die de naam van de klasse,
gevolgd door het suffix \lstinline !ID!.
\item Property's hebben dezelfde naam als de members (zonder eventueel prefix) die ze `accessen', maar dan met een
hoofdletter.
\item Namen van methods beginnen met een hoofdletter, en zijn van de vorm `(lijdend voorwerp +)
infitnitief'.\footnote{Op die manier worden methods die op hetzelfde betrekking hebben
door `intellisense' gegroepeerd.}  `Get' en `Set' kunnen in het Engels blijven.
\item Argumenten van een method krijgen namen met een kleine letter.
\end{enumerate}

\begin{lstlisting}[float,caption=Naamgeving klasses]
public class Cursus
{
	public void Annuleer()
	{
		// Doe vanalles...
	}

	public void DeelnemerInschrijven(Deelnemer inschrijver)
	{
		// Boeiende code weggelaten...
	}

	public decimal CursusID
	{
		get
		{
			return cursusID;
		}
	}

	private int cursusID;
};
\end{lstlisting}

\subsection{Speciale klasses}

\begin{enumerate}[resume]
\item Namen van exceptieklasses krijgen het suffix `\lstinline !Exception!'.
\item Namen van interfaces krijgen het prefix `\lstinline !I!'.
\item Namen van attribuutklasses krijgen het suffix `\lstinline !Attribute!'.
\item Arrays en collecties hebben een naam in het meervoud.
\end{enumerate}

\subsection{Overig}

\begin{enumerate}[resume]
\item Namen van \lstinline !namespace!s beginnen met een hoofdletter.
\item Zowel naam als waarden van een \lstinline !enum! krijgen een hoofdletter.
\item Namen van \lstinline !struct!s beginnen met een hoofdletter.
\item Lokale variabelen beginnen met een kleine letter.
\item Vermijd variabelenamen van 1 letter.
\end{enumerate}

\section{Declaraties}

\subsection{Algemeen}

\begin{enumerate}[resume]
\item Gebruik kleine letters voor de system data types
\begin{itemize}
\item \lstinline !int!
\item \lstinline !bool!
\item \lstinline !string!
\item \ldots
\end{itemize}
\item E\'en variabeledeclaratie per lijn.
\end{enumerate}

\begin{lstlisting}[float,caption=Declaraties algemeen]
string Naam;
string Voornaam;  // een nieuwe lijn voor een nieuwe declaratie
\end{lstlisting}

\subsection{Klasses}

\begin{enumerate}[resume]
\item Klasses die niet zichtbaar mogen zijn buiten de assembly, krijgen de \lstinline !internal!
access modifier.
\item Klasses met statische members hebben een statische default constructor.
\item Klasses met enkel statische members krijgen een \lstinline !private! constructor.
\item Ieder class member heeft een access modifier.\footnote{\lstinline !public!, \lstinline
!protected!, \lstinline !private!, \ldots}
\item In een klassedeclaratie zet je de member variabelen vooraan.\footnote{Op die manier is het direct duidelijk of de klasse stateful of stateless
is.}
\item Member variabelen zijn altijd \lstinline !private! of \lstinline !protected!.  Als ze van
buitenaf lees- of schrijfbaar moeten zijn, dan moet dit via property's gebeuren.
\item Gebruik geen member variabelen voor iets dat je ook in een lokale variabele kwijt kan.
\end{enumerate}

\subsection{Methods}

\begin{enumerate}[resume]
\item Niet meer dan 5 argumenten voor een method.  Maak zo nodig een \lstinline !struct! als
er meer informatie doorgespeeld moet worden.
\item Niet meer dan 200 lijnen code in \'e\'en method.
\end{enumerate}

\section{Implementatie}


\subsection{Assignaties}

\begin{enumerate}[resume]
\item Gebruik geen `magic numbers', maar \lstinline !enum!s of \lstinline !const!s.  (Zie listing
\ref{lstMagNum}.)
\begin{lstlisting}[float,caption=Geen magic numbers,label=lstMagNum]
// FOUT:
if (lid.HeeftFunctie(170))
{
	// doe iets
}

// GOED:
if (lid.HeeftFunctie(Functie.Vb))
{
	// doe iets
}
\end{lstlisting}
\item Geen letterlijke strings in programmacode.  Declareer ze ergens apart als constante, of
gebruik een resource file.  (Zie listing \ref{lstStrLit}.)
\begin{lstlisting}[float,caption=Geen letterlijke strings,label=lstStrLit]
// FOUT:
Console.WriteLine("Bent u zeker dat u '{1}' wil "
                  + "verwijderen?", Naam);

// GOED:
Console.WriteLine(Teksten.BevestigVerwijderenPersoon
                  , Naam);
\end{lstlisting}
\item Gebruik de return value van een assignatie niet direct opnieuw.  (Zie listing
\ref{lstAssRet}.)
\begin{lstlisting}[float,caption=Geen hergebruik return value van assignatie,label=lstAssRet]
// FOUT want zeer verwarrend:
if (i = j)
{
	// Deze code wordt enkel uitgevoerd als 
	// $j$ verschilt van 0.
}

// GOED:
i = j;
if (j != 0)
{
	// enz.
}
\end{lstlisting}
\item Vermijd het gebruik van \lstinline !null!.  Gebruik \lstinline !String.Empty!, \lstinline
!List<T>.Empty!, \ldots om een `leeg' resultaat voor te stellen.
\item Gebruik generics in plaats van instanties van de klasse `\lstinline
!Object!'.\footnote{Uitgezonderd bij het locken voor threading.}
\end{enumerate}

\subsection{Strings}

\begin{enumerate}[resume]
\item Gebruik de klasse `\lstinline !StringBuilder!' als je een groot aantal stringmanipulaties
moet uitvoeren op performantiekritsche plaatsen.
\item Gebruik `\lstinline !String.Compare!' om strings te vergelijken.
\item Stel de lege string voor door `\lstinline !String.Empty!' (en niet door \lstinline !null!).
\item Gebruik het prefix `@' voor een letterlijke string, in plaats van te escapen.
Bijvoorbeeld: \lstinline !@"Backslashes (\) zijn cool."!
\end{enumerate}

\subsection{Loops}

\begin{enumerate}[resume]
\item Roep dezelfde property niet herhaaldelijk aan.
\item Vermijd `dure' operaties in loops.
\item Vermijd recursieve functie-aanroepen waar mogelijk.
\end{enumerate}

\subsection{Arrays en collecties}

\begin{enumerate}[resume]
\item Gebruik waar mogelijk (begrensd aantal elementen met volgnummer) arrays.\footnote{Arrays zijn sneller dan collecties}
\item Gebruik de generieke collecties uit \lstinline !System.Collections.Generic!.
\item Gebruik de juiste collectie voor het juiste doel

\begin{tabular}{|c|l|}
\hline
elementen met index & \lstinline !List<T>! \\
zoeken mogelijk & \lstinline !Dictionary<TKey, TValue>! \\
gesorteerd & \lstinline !SortedDictionary<TKey, TValue>! \\
FIFO & \lstinline !Queue<T>! \\
LIFO & \lstinline !Stack<T>! \\
\hline
\end{tabular}
\end{enumerate}

\subsection{ASP.NET}

\begin{enumerate}[resume]
\item Vermijd het gebruik van sessievariabelen.
\item Genereer correct XHTML 1.1 (strict).
\item Gebruik style sheets (css) om de look en feel van een pagina te defini\"eren.
\item Gebruik nooit user input in een client side javascript.\opmerking{en plaats ook nooit
niet-gevalideerde userinput in een gegenereerde webpagina.}
\end{enumerate}

\subsection{Exception handling}

\begin{enumerate}[resume]
\item Vermijd excepties door op voorhand checks in te bouwen in de code.  Enkel om excepties op
te vangen waarop je geen vat hebt, gebruik je \lstinline !try! \ldots \lstinline
!catch!.\footnote{Dit is typisch het geval als je externe hardware of het netwerk aanspreekt.
In deze situaties moet je exception handling implementeren.}\footnote{Key-violations moet
je enkel op voorhand opvangen als je de gebruiker een zinvolle foutmelding kan geven.}
\item Maak zo weinig mogelijk gebruik van custom exceptions.  Als je er toch implementeert, erf
dan van `\lstinline !ApplicationException!', en maak ze \lstinline ![Serializable]!.
\item Als er zich een exceptie voordoet, geef je een verstaanbare boodschap voor de gebruiker,
en log je de technische details.
\item Catch enkel specifieke excepties.\footnote{Laat het programma gerust crashen op onverwachte
excepties, zo zie je ze bij het debuggen zeker niet over het hoofd.  In gereleaste versies mag
je generieke excepties opvangen op het hoogste niveau.}
\item Om in een \lstinline !catch!-blok de `gecatchte' exceptie opnieuw te `throwen', gebruik je
\lstinline !throw! zonder argumenten.\footnote{Op die manier wordt de volledige stack trace
behouden.}
\item Beperk de grootte van \lstinline !try! \ldots \lstinline !catch! structuren.  Liever
meerdere blokken dan \'e\'en groot blok.\footnote{Op die manier is het duidelijker in welk deel
van de code een exceptie optreedt.}
\end{enumerate}

\begin{lstlisting}[float,caption=Exception handling]
try
{
	// lees uit bestand
}
catch (FileIOException ex)
{
	// catch enkel specifieke excepties
	// log error
	// geef duidelijke informatie aan gebruiker
	// throw zo nodig exceptie opnieuw:

	throw;
}
finally
{
	// `dispose' bestand
}
\end{lstlisting}

\subsection{Garbage collection}

\begin{enumerate}[resume]
\item Implementeer de interface `\lstinline !IDisposable!' als je externe resources (connecties,
files) gebruikt.
\item Zorg ervoor dat de method `\lstinline !Dispose!' aangeroepen wordt van ieder disposable
object dat je gebruikt.
\item Gebruik geen unsafe code.
\item Roep de garbage collector zelf niet expliciet aan.
\end{enumerate}

\subsection{ADO.NET}\opmerking{Ik heb deze sectie laten staan, maar vermoedelijk gaan we LINQ to
SQL en Entity Classes gebruiken i.p.v. ADO.NET}

\begin{enumerate}[resume]
\item Connecties zo laat mogelijk openen en zo vroeg mogelijk opnieuw sluiten.\footnote{.NET is
zelf verantwoordelijk voor connection pooling.}
\item Zorg ervoor dat de connectie altijd gesloten wordt, ofwel in een \lstinline!finally!-blok, ofwel via een \lstinline !using!-constructie.
\item Gebruik een `Data Provider Factory' om connecties, commando's, \ldots te
genereren.\opmerking{Komt niet voor in Tommy's docs, maar lijkt me interessant als geen LINQ
gebruikt wordt.}
\item Gebruik de klasse \lstinline !DbParameter! om query's te parameterizeren.
\item Bewaar connection strings en data provider in het configuratiebestand.
\item Bij excepties log je de volgende zaken:
\begin{itemize}
\item Naam van het commando
\item Het commando zelf
\item De parameters
\item De connection string
\end{itemize}
\end{enumerate}

\begin{lstlisting}[float, caption=ADO.NET]
try
{
	// factory is de provider factory.

	DbConnection connectie = factory.CreateConnection();
	connectie.ConnectionString = connStr;

	DbCommand commando = factory.CreateCommand();
	commando.Connection = connectie;
	commando.CommandText = InschrijvingsQuery;
	
	DbParameter parameter = factory.CreateParameter();
	parameter.ParameterName = "@AdNr";
	parameter.Value = adNr;

	commando.Parameters.Add(parameter);
	commando.ExecuteNonQuery();
}
// handel hier exceptions af
finally
{
	commando.Dispose();
	connectie.Dispose();
}
\end{lstlisting}

\subsection{Overig}

\begin{enumerate}[resume]
\item Lock enkel objecten van het type `\lstinline !Object!'
\item Event handlers zijn kort, en roepen andere methods op.
\item Event handlers worden enkel gebruikt om hun event(s) af te handelen, en worden voor de rest
niet aangeroepen.
\item Gebruik liever geen late binding.
\end{enumerate}

\section{Gedrag van de software}

\begin{enumerate}[resume]
\item Geef steeds verstaanbare foutmeldingen voor de gebruiker.
\item Voer deze checks uit bij het starten van het programma:
\begin{enumerate}
\item Zijn alle files/resources/connecties beschikbaar?
\item Bestaat het configuratiebestand? Zo nee: maak het aan met standaardwaarden.
\item Is het configuratiebestand correct?
\end{enumerate}
\item Ga er niet vanuit dat het programma op een bepaalde schijf of in een bepaalde directory
ge\"installeerd is.
\item Gebruik `isolated storage' als je gebruikersspecifieke informatie moet bewaren.
\end{enumerate}

\end{document}
