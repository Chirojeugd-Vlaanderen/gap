\documentclass[a4paper,11pt]{article}
\usepackage[dutch]{babel}
\usepackage[pdftex]{graphicx}
\usepackage{multirow}
\usepackage[absolute]{textpos}
\usepackage{enumitem}
\usepackage{listings}
\usepackage{makeidx}
\lstloadlanguages{[Sharp]C}

\hyphenation{wij-zi-ging}
\hyphenation{be-kij-ken}
\hyphenation{be-re-ke-ning}


% dit document zou moeten compileren met texlive-latex-base
% en texlive-latex-extra geïnstalleerd

\newcommand{\opmerking}[1]{\marginpar{\scriptsize #1}}
\newcommand{\oeps}{\textbf{OEPS! }}

\setlength{\TPHorizModule}{25mm}
\setlength{\TPVertModule}{\TPHorizModule}
\textblockorigin{15mm}{15mm}

% Sans-serif font, volgens huisstijl
%
\renewcommand{\familydefault}{\sfdefault}

\makeindex

\begin{document}

%% algemene instellingen voor listings
\lstset{language=[Sharp]C, tabsize=3, frame=single, basicstyle=\small, texcl=true}

%%
%% Eerst een blok rommel voor de 'huisstijlheader'
%%

\begin{textblock}{2}(0,0)
\includegraphics[height=32mm]{chiro-grijs.jpg} \\
\end{textblock}

\begin{textblock}{4.1}(3,0)
\textbf
{
\begin{tabular}{|p{65mm}|p{10mm}|r|}
\multicolumn{3}{r}{\Large Chirojeugd-Vlaanderen vzw} \\
\multicolumn{3}{r}{\,} \\
\hline
\multicolumn{2}{|l|}{Werkgroep Informatica} & 271 \\
\hline
25 april 2008 & Doc. & 2 \\
\hline
\end{tabular} 
}
\end{textblock}

\vspace*{4mm}

%%
%% HIER BEGINT HET DOCUMENT
%%


\noindent {\Large \bf CODING STANDARD C\#, doc. v. 0.7}

\tableofcontents

\section{Over dit document}

\subsection{Opmerkingen vooraf}

Dit document is opgesteld vooraleer we begonnen met programmeerwerk.  Wat
betreft indentatie en maximale lijnlengte, is het niet meer helemaal up to
date.  In praktijk gebruiken we voor indentatie en maximale lijnlengte
de standaardinstellingen van Visual Studio en Resharper 6.  (Voor
Resharper zitten de configuratiebestanden bij in de repo.)

\subsection{Doel van het document}

Dit document geeft een aantal richtlijnen voor de ontwikkeling van
C\#.Net-applicaties, zowel over opmaak en naamgeving als over de manier van
dingen implementeren.  Deze coding guidelines streven het volgende na:
\begin{itemize}
\item leesbaarheid en het onderhoudsgemak dankzij een uniforme stijl
\item veilige code
\item de tools in de IDE\footnote{code completion, automatisch
gegenereerde code, \ldots} optimaal gebruiken
\item bugs vermijden via duidelijk omschreven `patterns'
\end{itemize}
Het document is opgesteld naar aanleiding van de ontwikkeling van de
opvolger van het Chirogroepprogramma.
De bedoeling is dat de richtlijnen in dit document (en zijn opvolgers) toegepast zullen worden
in alle C\#-projecten die nog voor de Chiro ontwikkeld zullen worden.

\subsection{Aanpassingen aan dit document}

Voorstellen voor wijzigingen en aanvullingen aan dit document worden via
e-mail verstuurd naar de werkgroep Chirogroepprogramma.  In de
`managementvergaderingen' voor Chirogroep van december, april en augustus
wordt beslist of voorgestelde wijzigingen al dan niet doorgevoerd worden.

Als er een richtlijn wegvalt, blijft het volgnummer van die richtlijn
ongebruikt, zodat verwijzingen naar bestaande nummers correct blijven.
Enkel bij een grote wijzigingen van de coding standard kan er
hernummerd worden.

\section{Architectuur}

\begin{enumerate}
\item We werken met een lagenmodel\index{lagenmodel}; iedere laag staat op zich, en communiceert slechts met de
andere lagen via publieke klasses/interfaces.  We implementeren voor elk
project deze lagen:
\begin{itemize}
\item Backend
\begin{itemize}
\item Orm -- alles wat met de Object Relational Mapping (ORM) te
maken heeft: de gegenereerde klasses, de uitbreidingen in partial classes,
en de interfaces voor de Data Access Objects (DAO's)
\item Data -- De Data Access Objects (repository's)
\item Workers -- De businesslogica: Worker classes waarvan de methods zo
veel mogelijk met objecten werken.
\item ServiceContracts -- Interfaces voor WCF-services (servicecontracts)
en datacontracts.  De services exposen geen entity's uit 
`Orm'.\index{datacontracts}\index{servicecontracts}
\item Services -- WCF-services.  De services doen niet veel meer dan
objecten ophalen, en daar den de workers op loslaten.
\end{itemize}
\item Frontend (user interface)
\item `Domain' (enums en interfaces zowel van toepassing op backend als
op frontend)
\end{itemize}
\item Data Access Objects worden ge\"instantieerd via de Unity
IoC-container.\index{unity}
\item We gebruiken Entity Framework voor ORM\index{entity framework}
\item [3-b.] We gebruiken ASP.NET MVC voor webapplicaties\index{MVC}
\item [3-c.] We gebruiken JQuery als framework voor AJAX\index{JQuery}\index{AJAX}
\item We gebruiken subversion voor source control en trac voor issue
tracking.\index{subversion}\index{trac}
\end{enumerate}

\section{Structuur}

\subsection{Bronbestanden}
\index{bronbestanden}

\begin{enumerate}[resume]
\item E\'en klasse, struct\index{struct} of interface per bestand, met inner classes als enige uitzondering.
\item De bestandsnaam\index{bronbestanden!naam} moet overeen komen met de naam van de klasse, struct
of interface.
\item Generieke klasses \index{generics}\index{generics!klasses} komen in een bestand waarvan de naam het suffix
\_{\em n}' krijgt, waarbij {\em n} het aantal generieke parameters
is.  Bijv. `IDao\_1'.
\item Als een \lstinline !enum!\index{enum} slechts voor \'e\'en klasse gebruikt
wordt, kan die \lstinline !enum! in hetzelfde bronbestand als de klasse
gedefinieerd worden.  Alle andere \lstinline !enum!s worden per
assembly/namespace gegroepeerd in een bestand Enums.cs\index{Enums.cs}.

\end{enumerate}

\subsection{Assembly's}
\index{assembly's}

\begin{enumerate}[resume]
\item Assembly's hebben dezelfde naam als hun namespace.
\item Specifieer bedrijf, copyright en versienummer in de AssemblyInfo.cs\index{AssemblyInfo.cs} van
elke toepassing.
\item Mix geen third-party code\index{third-party code} met eigen code in dezelfde assembly.
\item Unit tests\index{unit tests} komen in \'e\'en of meerdere aparte
assembly's.
\end{enumerate}

\subsection{Namespaces}
\index{namespaces}

\begin{enumerate}[resume]
\item Gebruik nooit de globale namespace.
\item Gebruik geen `gereserveerde' namespaces zoals \lstinline !System!.
\item Gebruik geen fully-qualified type names, maar \lstinline !using!.
\item Als er op die manier conflicten ontstaan, maak dan een alias d.m.v.
\lstinline !using!.
\end{enumerate}

\subsection{Gegenereerde code}
\index{gegenereerde code}

\begin{enumerate}[resume]
\item Wijzig geen third-party code\index{third-party code} of gegenereerde code;  Gebruik hiervoor
partial classes\index{partial classes}.
\item Geen applicatielogica in \lstinline !AssemblyInfo.cs!\index{AssemblyInfo.cs}.
\end{enumerate}

\subsection{Bestandsopbouw}
\index{bronbestanden!opbouw}
Zie listing \ref{lblLstOpbouw} voor voorbeelden.

\begin{enumerate}[resume]
\item Elk bronbestand\index{bronbestanden} begint met eenzelfde blokje
commentaar, zie listing \ref{lblLstOpbouw}.\opmerking{De exacte tekst
ligt nog niet vast.}\footnote{We zullen hiervoor een template voorzien
voor Visual Studio.}
\item Groepeer de imports van namespaces bovenaan\index{namespaces}\index{using}
\begin{itemize}
\item Eerst de .Net namespaces
\item Dan 3rd-party namespaces
\item Tenslotte de eigen namespaces
\end{itemize}
\item Hoogstens \'e\'en namespace in \'e\'en bronbestand.
\item Gebruik \lstinline !#region!\index{region@\#region} \ldots \lstinline !#endregion!-constructies om stukken code
die bij elkaar horen te groeperen.
\item Bronbestanden mogen niet groter worden\index{bronbestanden!grootte} dan 1000 lijnen code, omdat grote
files de
Visual Studio Debugger vertragen.  Splits de code van grotere klasses op in
`logische' stukken, maak er partial classes\index{partial classes}\index{klasses!partial} van.  Geef ze namen zoals
`KlasseNaam.OmschrijvingStukCode.cs'.
\item Methods\index{methods!grootte} mogen niet langer worden dan 200 lijnen code.
\end{enumerate}

\subsection{Line breaking}
\begin{enumerate}[resume]
\item Vermijd lijnen\index{lijnen!lengte} die langer zijn dan 120 karakters.
\item Onderbreek lijnen\index{lijnen!line breaking} zo weinig mogelijk, en enkel indien nodig.
\item Na een line break volgt een indentatie.\index{indentatie}
\item De meest natuurlijke plaats voor een line break is na de
assignatie-operator.
\begin{lstlisting}
GelieerdePersonenManager mgr =
	Factory.Maak<GelieerdePersonenManager>();
\end{lstlisting}
\item Als dat niet volstaat, breek dan voor een lijst, na het haakje:
\begin{lstlisting}
public GroepenManager(
	IGroepenDao grpDao, IDao<AfdelingsJaar> afdjDao)
\end{lstlisting}
\item Tussen parameters mogen line breaks komen, maar dan moet iedere
parameter op een nieuwe lijn.  Breek na de komma, dan werkt Visual Studio
beter mee.
\begin{lstlisting}
db.AttachObjectGraph(
	afdelingsJaar,
	aj => aj.OfficieleAfdeling.WithoutUpdate(),
	aj => aj.Afdeling.WithoutUpdate(),
	aj => aj.GroepsWerkJaar.WithoutUpdate());
\end{lstlisting}
\item Als er een logische connectie bestaat tussen opeenvolgende parameters,
dan mag je ze bij line-breaking wel op dezelfde lijn laten staan.
\begin{lstlisting}
DoeIetsMetGegevens(
	straat, nr, bus,
	postnr, gemeente,
	land);
\end{lstlisting}
\item Gebruik een dubbele indentatie als je bovenstaande line-breakings
combineert.
\begin{lstlisting}
var gelieerdePersonen =
	pm.PaginaOphalenMetLidInfo(
		groepID,
		pagina,
		paginaGrootte,
		out aantalTotaal);
\end{lstlisting}
\end{enumerate}

\begin{lstlisting}[float, caption=Opbouw, label=lblLstOpbouw]
// Contacteer informatica@chiro.be voor alle info over deze broncode.
//

using System;
using System.Data;
using System.Data.Linq.Mapping;
using KipAdmin.StamNrTools;

namespace Chiro.CG2
{
	/// \textless summary \textgreater
	/// Basisklasse voor ChiroGroep en Satelliet
	/// \textless /summary \textgreater
	public class Groep
	{
		// ...
	}
}
\end{lstlisting}

\subsection{Whitespace en indentatie}
\index{indentatie}
\index{whitespace}

Voorbeelden in listing \ref{lblLstWhitespace}.

\begin{enumerate}[resume]
\item Geen spaties na een (rond) haakje-open of voor een (rond) haakje-toe\index{haakjes}
\item E\'en spatie tussen \lstinline !if!\index{if}, \lstinline !for!\index{for},
\lstinline !while!\index{while} en het haakje-open.  Bijv.: \lstinline !if (voorwaarde)!
\item Een spatie voor en na alle infix-operatoren\index{operatoren!infix}, inclusief de \lstinline
!=>! van lambda-expressies\index{lambda-expressies}.
\item Geen spatie tussen een prefixoperator\index{operatoren!prefix} (bijv. `\lstinline !-!' of
`!') en zijn argument.
\item Scheid blokken gerelateerde code van elkaar door \'e\'en lege lijn\index{lijnen!lege}.
\item Gebruik tabs\index{tab} voor indentatie\index{indentatie}
\item Indenteer ook commentaar\index{commentaar}
\item Voor `blokken' code tussen accolades\index{accolades}, zijn de accolades ge\"indenteerd op het niveau van
de bovenliggende structuur.  Op een regel met een accolade staat enkel die accolade.  Wat binnen de accolades staat, is ge\"indenteerd.
\item Uitzonderingen op bovenstaande regel:
\begin{itemize}
\item Eenvoudige getters en setters worden op \'e\'en lijn gedefinieerd, inclusief accolades.
\item Abstracte\index{property's!abstracte} en automatische\index{property's!automatische} property's worden op \'e\'en lijn
gedeclareerd, inclusief accolades.
\item Lege functies (bijv. constructors\index{constructors}) worden op \'e\'en lijn
gedefinieerd, inclusief accolades, bijv. 
\lstinline !EenKlasse(string naam): base(naam) { }!.
\end{itemize}
\item Geen spatie tussen de naam van een method en het haakje daarna.  Wel een spatie na de komma's\index{komma's} tussen verschillende
parameters\index{parameters!white space}\index{methods!parameters}.  

Bijv.: \lstinline
!DoeIets(argument, tweedeArgument)!.
\end{enumerate}

\begin{lstlisting}[float, caption=Indentatie, label=lblLstWhitespace]
public class Wereld: Planeet
{
	public Wereld(string naam, IZon zon): base(naam, zon) { }

	public void Hallo(string[] groeters)
	{
		foreach (g in groeters)
		{
			// Commentaar ge\"indenteerd

			Console.WriteLine(String.Format(
				"{0}: Hallo {1}!", g, Naam));
		}
	}	
}
\end{lstlisting}


\subsection{Commentaar}
\index{commentaar}

\begin{enumerate}[resume]
\item Commentaar staat steeds v\'o\'or het stuk code waarop het betrekking
heeft, met als mogelijke uitzondering commentaar van \'e\'en lijn betreffende \'e\'en
lijn code: die mag op diezelfde lijn code.
\item Zet in methods\index{methods!commentaar} enkel commentaar waar het nodig is, om de werking van je code te
verduidelijken.  Als je leesbare code schrijft, kan je de hoeveelheid commentaar beperken.
Complexe code moet goed gedocumenteerd zijn.
\item Gebruik \lstinline !//! of \lstinline !///! (voor XML-documentatie, zie sectie
\ref{sectDocumentatie}) voor `structureel'
commentaar in je code.  \lstinline !/*! en \lstinline !*/! kan je dan nog gebruiken om tijdelijk
een stuk code uit te commentari\"eren.
\item Uitgecommentarieerde code mag niet gecommit worden in source control\index{source control}.
\item Schrijf commentaar in kort, duidelijk en correct Nederlands :)
\item Gebruik 'TODO:'\index{TODO} om aan te geven dat er nog iets moet gebeuren.  Als
zo'n TODO gecommit wordt naar de source control, moet er een
ticket\index{ticket}\index{trac} voor aangemaakt zijn in de track, en moet de TODO: naar het
ticketnummer verwijzen.
\item Code waarvan de bedoeling niet direct duidelijk is, maar die werd
geschreven om een bug te fixen\index{bugfix}, krijgt in commentaar een link naar de
bug in de Trac.
\item Commentarieer veronderstellingen die niet direct duidelijk zijn bij
het bekijken van de code.
\end{enumerate}

\subsection{Documentatie}
\label{sectDocumentatie}
\index{documentatie}

\begin{enumerate}[resume]
\item Plaats een blok XML-documentatie voor
\begin{itemize}
\item alle methods\index{methods!documentatie}\footnote{voor code completion\index{code completion}}
\item alwat \lstinline !public!\index{public!documentatie} of \lstinline !protected!\index{protected!documentatie} is.
\end{itemize}
\item Geen XML-documentatie voor vanzelfsprekende \lstinline !private!\index{private!documentatie} en
\lstinline !internal!\index{internal!documentatie} code.
\item Guidelines voor XML-documentatie:
\begin{itemize}
\item Wees consistent\index{consistentie} als je gelijkaardige members\index{class members} documenteert.
\item Als je de implementatie van een interface\index{interfaces!documentatie} documenteert, kopieer dan
de documentatie van de interface, en vul eventueel aan met
implementatiespecifieke info.
\item Gebruik \lstinline !<see>!\index{see@\textless see\textgreater}-tags om naar property's\index{property's!documentatie}, methods\index{methods!documentatie} en
klasses\index{klasses!documentatie} te verwijzen.
\item Gebruik \lstinline !<paramref />!\index{paramref@\textless paramref\textgreater} en
\lstinline !<typeparamref />!\index{typeparamref@\textless
typeparamref\textgreater} om
naar parameters\index{parameters!documentatie} te verwijzen.
\item Gebruik \lstinline !<c>!\index{c@\textless c\textgreater}-tags voor code-elementen.
\item Gebruik steeds \lstinline !<c>!-tags voor `\lstinline !null!'\index{null!documenatie},
`\lstinline !false!'\index{false!documentatie} en `\lstinline !true!''\index{true!documentatie}.
\item Gebruik zowel de \lstinline !<summary>!\index{summary@\textless summary\textgreater}- als de \lstinline
!<value>!\index{value@\textless value\textgreater}-tag.\footnote{Ja, dit wil zeggen dat je tekst moet herhalen.}
\item Als een parameter niet \lstinline !null! mag zijn, moet dat zo
gedocumenteerd zijn.
\item Standaardvoorbeelden vind je in listing \ref{lstDocumentatie}
\end{itemize}
\item Documentatie voor frameworks\index{frameworks!documentatie} moet zeer uitgebreid zijn.
\item Voorzie tutorials\index{tutorials} en
voorbeeldcode\index{voorbeelden} op de wiki voor gebruikte frameworks en
patterns\index{patterns}.
\end{enumerate}

\begin{lstlisting}[float,caption=XML-documentatie, label=lstDocumentatie]
/// \textless summary\textgreater
/// Ophalen van de unity container.
/// \textless summary\textgreater
/// \textless value\textgreater De unity container.\textless value\textgreater
protected static IUnityContainer Container { get; }

/// \textless summary\textgreater
/// Ophalen of bewaren van het geslacht.
/// \textless summary\textgreater
/// \textless value\textgreater Het geslacht.\textless value\textgreater
public GeslachtsType Geslacht { get; set; }

/// \textless summary\textgreater
/// Ophalen of bewaren van een waarde die bepaalt of 
/// de entiteit verwijderd moet worden.
/// \textless summary\textgreater
/// \textless value\textgreater
/// \textless c\textgreater true\textless c\textgreater als de entiteit verwijderd moet worden,
/// anders \textless c\textgreater false\textless c\textgreater
/// \textless value\textgreater
bool TeVerwijderen { get; set; }

/// \textless summary\textgreater
/// Koppelt het adres \textless paramref name=``adr'' /\textgreater aan de 
/// gelieerde persoon \textless paramref name=``pers'' /\textgreater
/// \textless summary\textgreater
/// \textless param name=``pers''\textgreater
/// Te koppelen gelieerde persoon; mag niet \textless c\textgreater null\textless c\textgreater zijn.
/// \textless param\textgreater
/// \textless param name=``adr"\textgreater
/// Te koppelen adres; mag niet \textless c\textgreater null\textless c\textgreater zijn.
/// \textless param\textgreater
void AdresKoppelen(GelieerdePersoon pers, Adres adr)
\end{lstlisting}

\section{Naamgeving}
\index{namen}

\subsection{Namen kiezen}

\begin{enumerate}[resume]
\item Alle eigen gekozen namen zijn in het Nederlands. 
\item Namen zijn zo kort mogelijk, maar moeten wel een duidelijke
betekenis hebben.
\item Vermijd variabelenamen van 1 letter.
\item Instanties van arrays\index{arrays!namen} en collecties\index{collecties!namen} hebben een naam in het meervoud.  Gebruik geen
suffix \lstinline !Lijst! of dergelijk.
\item Underscores \index{underscores} zijn enkel toegelaten in publieke constanten en als (enkelvoudige) prefix van een member\index{membervariabelen!namen} variable.
\item Gebruik geen prefixnotaties zoals Hongaarse notatie\index{Hongaarse notatie}.
\item Vermijd het gebruik van cijfers\index{cijfers} in variabelenamen, tenzij
\begin{itemize}
\item als suffix voor namen van lokale variabelen\index{lokale variabelen!cijfers}
en methodparameters
\item als ze een betekenis hebben, zoals bijv. in `\lstinline !A4Formaat!'.
\end{itemize}
\item Indien een naam uit meerdere woorden bestaat, worden de woorden aan elkaar geschreven,
waarbij elk nieuw woord met een hoofdletter begint.  Ook in samenstellingen\index{samenstellingen} start elk
`deelwoord' met een hoofdletter.\footnote{Deze laatste regel is niet absoluut; zo kan bijv. in
\lstinline !leeftijd! de `t' een kleine letter blijven, hoewel dit een samenstelling is.  Het is moeilijk om hier een regel op te
plakken.}
\begin{itemize}
\item \lstinline !geboorteDatum!
\item \lstinline !groepsLeiding!
\end{itemize}
\item Gebruik liever geen afkortingen\index{afkortingen}, tenzij ze door alle developers
gekend zijn.
\item Gebruik enkel letterwoorden\index{letterwoorden} die algemeen aanvaard zijn.
\item Voor letterwoorden en afkortingen gelden dezelfde regels als voor
`gewone woorden'.  Dus geen `all caps'.
\begin{itemize}
\item \lstinline !isVzw!
\item \lstinline !Gap.Ioc!
\item \lstinline !AdNummer!
\end{itemize}
\item Uitzonderingen op vorige regel zijn \lstinline !ID! en 
\lstinline !IO!.
\item Identifi\"ers die enkel qua `casing' verschillen zijn enkel
toegelaten als ze naar verschillende language elements verwijzen (bijv.
een property en een lokale variabele zoals in 
\lstinline !Leeftijd = leeftijd!.)
\end{enumerate}

\subsection{Types}
\index{types!namen}

\begin{enumerate}[resume]
\item Geen `libraryprefixes\index{libraryprefixes}'\footnote{Een `libraryprefix' is een prefix
gebruikt voor alle types in een assembly/library.}
\item Vermijd te vage typenames\index{types!namen} zoals `\lstinline !Element!',
`\lstinline !Log!', \ldots om conflicten te vermijden.
\item Verschillende implementaties van eenzelfde concept krijgen een
gemeenschappelijk suffix\index{consistentie}, zoals `\lstinline !PersonenDao!', `\lstinline !GroepenDao!', \ldots
\end{enumerate}

\subsection{Klasses algemeen}
\index{klasses!namen}

\begin{enumerate}[resume]
\item Namen van klasses zijn in het enkelvoud, en beginnen met een hoofdletter.
\item Als een klasse slechts \'e\'en interface\index{interfaces!implementatie} implementeert, dan moet de
klassenaam de naam van de interface bevatten.  Zie bijv. listing \ref{lstKlasseNaamgeving}.
\item Statische klasses\index{klasses!statische}\index{statische klasses} die gebruikt worden als helper\index{helper}, krijgen het suffix
`\lstinline !Helper!', bijv. \lstinline !EntityFrameworkHelper!.
\item [74-b.] Datacontracts die ge\"exposed worden door een service, krijgen suffix `\lstinline !Info!' of `\lstinline !Detail!', alnaargelang het om een beperkt of een uitgebreid datacontract gaat.\index{datacontracts}\index{info}\index{detail}
\end{enumerate}


\begin{lstlisting}[float,caption=Naamgeving klasses,label=lstKlasseNaamgeving]
public class AltijdGavAutorisatieDao: IAutorisatieDao
{
	private string _login;

        public bool IsGavGelieerdePersoon(
		int gelieerdePersoonID)
        {
            return true;
        }

	public string Login
	{
		get { return _login; }
	}

	// ...

};
\end{lstlisting}

\subsection{Specifieke klasses}

\begin{enumerate}[resume]
\item Namen van exceptieklasses\index{excepties!namen} krijgen het suffix `\lstinline !Exception!'.
\item Namen van interfaces\index{interfaces!namen} krijgen het prefix `\lstinline !I!'.
\item Namen van attribuutklasses\index{attributen!namen} krijgen het suffix `\lstinline !Attribute!'.
\end{enumerate}

\subsection{Property's}
\index{property's!namen}
\begin{enumerate}[resume]
\item Het achterliggend veld\index{property's!achterliggende velden}\index{membervariabelen!namen} (if any) van een property heeft dezelfde naam
als de property, maar met een kleine letter en geprefixt met een underscore\index{underscore}.
\item Geen type-informatie\index{type-informatie}\index{redundantie} in propertynamen.
\item Boolean\index{bool} property's krijgen het prefix \lstinline !Is! als dat de
leesbaarheid bevordert.
\item Maak de naam van een property zo kort mogelijk, zonder informatie te
verliezen.  \item Als twee property's op dezelfde manier een kortere naam kunnen
krijgen, gebruik dan voor geen van beide die kortere naam.
\item Herhaal geen informatie uit een klassenaam in de naam van een
property.  Als de klasse bijvoorbeeld `\lstinline !Categorie!' heet, kies
dan eerder voor een property `\lstinline !Naam!' in plaats van `\lstinline
!CategorieNaam!'.\index{redundantie}
\item Als een klasse een identificatienummer\index{ID}\index{identificatienummer}\index{property's!ID} (ID) heeft, voorzie dan een
property met de naam `\lstinline !ID!'.
\end{enumerate}

\subsection{Methods}
\index{methods!namen}
\begin{enumerate}[resume]
\item Namen van methods beginnen met een hoofdletter, en zijn van de vorm `(lijdend voorwerp\index{lijdend voorwerp} +)
infitnitief'.\footnote{Op die manier worden methods die op hetzelfde betrekking hebben
door `intellisense' gegroepeerd.}  `Get'\index{Get} en `Set'\index{Set} kunnen in het Engels blijven.
\item Herhaal geen informatie uit een klassenaam in de naam van een
method\index{redundantie!methodnamen}.  Als de interface bijvoorbeeld `\lstinline !IPersonenDao!' heet, kies
dan eerder voor een method `\lstinline !Ophalen!' in de plaats van `\lstinline
!PersoonOphalen!'.
\end{enumerate}

\subsection{Parameters}
\index{parameters}
\index{parameters!namen}
\begin{enumerate}[resume]
\item Method parameters krijgen namen met een kleine letter.
\item Liever volledige woorden dan afkortingen\index{afkortingen}.
\item Vermijd informatie uit het type te herhalen in de naam van de
parameter\index{redundantie!typenamen}.  Als bijv. de parameter van het type \lstinline !IDataContext!
is, dan kan de parameter eenvoudigweg \lstinline !context! heten.
\item Methodparameters in interface-implementaties\index{methods!interface-implementatie} hebben dezelfde naam\index{consistentie} als de
methodparameters in de interfaces.
\item Als twee of meer overloads een gelijkaardige parameter hebben, geef
die parameter\index{parameters!namen} dan dezelfde naam in iedere overload.
\item Als twee parameters op dezelfde manier verkort\index{verkorting} kunnen worden,
verkort dan geen van beide.
\end{enumerate}

\subsection{Lokale variabelen}
\begin{enumerate}[resume]
\item Namen van lokale variabelen\index{lokale variabelen!namen}\index{lokale variabelen} beginnen met een kleine letter.
\item Hoe beperkter de scope, des te meer namen van variabelen afgekort
mogen worden.
\item Binnen eenzelfde method mogen geen twee lokale variabelen voorkomen
met dezelfde naam (ook al zitten ze in een andere scope).\footnote{Op die
manier wordt verwarring vermeden als een scope verdwijnt.}
\end{enumerate}

\subsection{Generieke parameters}
\index{generics}\index{parameters!generiek}\index{generics!methods}
\begin{enumerate}[resume]
\item Als een method \'e\'en generieke parameter heeft, noem die parameter
dan `\lstinline !T!'.
\item Als een method twee generieke parameters heeft, en het gaat om keys
en values, gebruik dan `\lstinline !TKey!' en `\lstinline !TValue!'.
\item Indien van toepassing, mag de generieke parameter ook `\lstinline
!TResult!' heten.
\item Voor generieke conversies\index{conversies} kan je `\lstinline !TInput!' en `\lstinline
!TOutput!' gebruiken.
\item Als er meerdere generieke parameters zijn, en geen van bovenstaande
regels is van toepassing, dan begint de naam van elke generieke
parameter met de letter T.
\end{enumerate}

\subsection{Lambda-expressies}
\index{lambda-expressies}
\begin{enumerate}[resume]
\item Gebruik niet de zeer nietszeggende naam `\lstinline !x!' voor
parameters.
\item Volg dezelfde regels voor naamgeving van parameters in
lambda-expressies als voor parameters van standaardmethods.
\end{enumerate}

\subsection{Namespaces}
\index{namespaces}
\begin{enumerate}[resume]
\item Namen van \lstinline !namespace!s beginnen met een hoofdletter.
\item Geen namespace-hi\"erarchie\"en\index{namespaces!hi\"erarchie} van meer dan vier niveaus.
\item Geen versienummers\index{versienummers} in namen van namespaces.
\item Namespaces krijgen namen van concepten.
\item Namespaces starten met de naam van het bedrijf, gevolgd door de
naam van de applicatie, bijv. `\lstinline !Chiro.Gap.Xxx!'.
\end{enumerate}

\subsection{UI-elementen}
\index{UI}\index{user interface}
TODO.\footnote{Voor het Chirogroepprogramma is dit irrelevant, omdat we
MVC gebruiken.  Er zijn dus geen objecten gekoppeld aan de UI-elementen}

\subsection{Overig}

\begin{enumerate}[resume]
\item Zowel naam als waarden van een \lstinline
!enum!\index{enum}\index{enum!namen}\index{enum!waarden} beginnen met een hoofdletter.
\item Een \lstinline !enum!\index{enum!namen} krijgt een naam in het enkelvoud, tenzij de
enum als bitset wordt gebruikt.
\item Namen van \lstinline !struct!s\index{struct} beginnen met een hoofdletter.
\item Namen van \lstinline !const!s\index{const} zijn volledig in
hoofdletters, waarbij de `deelwoorden' van elkaar gescheiden worden met
een underscore.  Bijvoorbeeld \lstinline!MIJN_GROEP_ID!.
\item Namen van compiler variabelen\index{compiler!variabelen} zijn volledig in hoofdletters.
\item Resources hebben alfanumerieke namen\index{resources!namen}, maar mogen een punt (.)
bevatten om een soort van nesting\index{resources!nesting} te emuleren.
\end{enumerate}

\section{Implementatie}

\subsection{Declaraties algemeen}

\begin{enumerate}[resume]
\item Gebruik in declaraties kleine letters voor de system data types\index{system types}
\begin{itemize}
\item \lstinline !int!\index{int}
\item \lstinline !bool!\index{bool}
\item \lstinline !string!\index{string}
\item \ldots
\end{itemize}
\item Als je statische functies van system
types\index{system types!statische functies}\footnote{\lstinline !string!, \lstinline !int!, \ldots} gebruikt,
krijgen de types wel een hoofdletter.  Bijv. 
\lstinline !String.IsNullOrEmpty!\index{String.IsNullOrEmpty}.
\item E\'en variabeledeclaratie\index{variabelen!declaratie}\index{lokale variabelen!declaratie}\index{membervariabelen!declaratie} per lijn.
\item Gebruik voor declaraties waar mogelijk interfaces\index{interfaces}\index{interfaces!variabelendeclaraties}, zelfs voor lokale variabelen\index{lokale variabelen!declaratie} of
membervariabelen\index{membervariabelen!declaratie}. 
\item Elk type\index{types!access modifier}, method\index{methods!access modifier} of field\index{fields} heeft een access modifier\index{access modifiers}.\footnote{\lstinline !public!, \lstinline
!protected!, \lstinline !private!, \ldots}
\end{enumerate}

\begin{lstlisting}[float,caption=Declaraties algemeen]
string Naam;
string Voornaam;  // een nieuwe lijn voor een nieuwe declaratie

public enum Kleur
{
  Rood, Groen, Blauw
}
\end{lstlisting}

\subsection{Gebruik van `\lstinline !var!'}
\index{var}
\begin{enumerate}[resume]
\item Gebruik `\lstinline !var!' voor een variabeledeclaratie als het de
leesbaarheid van de code vergroot.
\item Gebruik geen `\lstinline !var!' voor een basistype zoals
\lstinline !int!\index{int} of \lstinline
!string!\index{string}.\footnote{Deze regel is niet zo dwingend; zij
die liever \lstinline !var! gebruiken, mogen dat toch doen.}
\item Hoe beperkter de scope, hoe meer het gebruik van \lstinline !var!
gerechtvaardigd is.
\end{enumerate}

\subsection{Controlestructuren}

\begin{enumerate}[resume]
\item Gebruik steeds accolades\index{accolades} bij een \lstinline !if!\index{if}, \lstinline !while!\index{while},
 \lstinline !for!\index{for},\ldots, ook als er maar 1 (of geen) statement moet staan.  Zie listing
\ref{lblLstControle}.
\begin{lstlisting}[float, caption=Controlestructuren, label=lblLstControle]
// FOUT want verwarrend:
for (i = 0; i < N; a[i++] = 0);
while (eenOfAnder());

// GOED:
for (i = 0; i < N; i++)
{
	a[i] = 0;
}
while (eenOfAnder())
{
	// doe niets
}
\end{lstlisting}
\item Gebruik geen \lstinline !goto!\index{goto}
\item Pas de loopvariabele van een \lstinline !for!-loop\index{for} niet aan binnen
de loop.
\item Loopvariabelen voor \lstinline !while!\index{while} worden ofwel aangepast aan het
begin van de loop, ofwel aan het einde.
\item Houd de body van een loop kort, en vermijd te diepe nesting\index{nesting!loops}.  Als de
body te complex wordt, gebruik dan liever een inner class\index{inner classes}\index{klasses!inner} of een private
method.\index{methods!private}\index{private!methods}
\item Roep dezelfde property niet herhaaldelijk aan.
\item Vermijd `dure' operaties in loops.
\item Vermijd recursieve\index{recursie} functie-aanroepen waar mogelijk.
\item Voorzie steeds een \lstinline !default!\index{default} case\index{case} bij een \lstinline
!switch!\index{switch}-constructie.  Als die situatie zich niet mag voordoen, zet daar
dan een \lstinline !Assert!\index{Assert} of \lstinline !throw!\index{throw}.  Zie
listing \ref{lblLstDefault}
\item Als in meerdere gevallen hetzelfde codeblok uitgevoerd moet worden,
groepeer dan deze \lstinline !case!s, zoals in listing \ref{lblLstDefault}.
\begin{lstlisting}[float, caption=switch statements', label=lblLstDefault]
switch (isnschrijver.Soort)
{
	case InschrijvingsType.Deelnemer:
	{
		// code voor deelnemers
		break;
	}
	case InschrijvingsType.Begeleider:
	case InschrijvingsType.KookPloeg:
	{
		// code voor begeleiding en kookploeg
		break;
	}
	default:
	{
		Debug.Assert(false, String.Format(
			"Ongeldige soort inschrijver: {0}",
			inschrijver.Soort));
	}
}
\end{lstlisting}
\item Voorzie steeds een \lstinline !else!\index{else} zonder voorwaarde in een \lstinline !if!\index{if} \ldots
\lstinline !else if! \ldots constructie.
\item Geen te moeilijke logica in een \lstinline !if!.  Als je \lstinline
!if! te complex wordt, gebruik een lokale \lstinline !bool!\index{bool}.
\item Een ternaire\index{operatoren!ternair}\index{?} (? :) of `coalescing\index{operatoren!coalescing}\index{??}' (??) operator mag enkel
gebruikt worden in statements die op \'e\'en lijn passen.  Converteer ze
naar een \lstinline !if!-structuur als dat niet lukt.
\end{enumerate}


\subsection{Enums}
\index{enum}\index{enum!declaratie}

\begin{enumerate}[resume]
\item Gebruik liever geen expliciete waarden voor \lstinline !enum!s.
\item Specifieer liever geen type\index{enum!type} voor een \lstinline !enum!, zodat het
standaardtype \lstinline !Int32! gebruikt wordt.
\item Gebruik \lstinline !enum!s in de plaats van een lijst statische constanten\index{constanten!static}, tenzij die lijst met constanten uitgebreid moet kunnen worden door ervende code.
\item Vermijd \lstinline !enums! te wijzigen\index{enum!wijzigen} als ze
ook gebruikt worden als veldwaarde in de database, of in code die je
niet zelf beheert.  Gebruik het attribuut \lstinline ![Obsolete]!\index{obsolete@[Obsolete]} om
members te markeren die niet meer gebruikt worden.
\item Assigneer geen expliciete waarden\index{enum!expliciete waarden} in een \lstinline !enum!, tenzij ze nodig zijn voor
persistentie in de database.
\item De eerste waarde in een \lstinline !enum! is de standaardwaarde\index{enum!standaardwaarde}.
\end{enumerate}

\subsection{Bitsets}
\index{bitsets}
\begin{enumerate}[resume]
\item Gebruik een \lstinline!enum! met het attribuut \lstinline ![Flags]!\index{flags@[Flags]} om een bitset
te defini\"eren.
\item Voor een bitset assigneer je de vlaggen\index{vlaggen} als machten van 2 in
hexadecimale notatie.
\item De eerste waarde van een bitset is altijd \lstinline !Geen!\index{Geen}, met als
waarde \lstinline !0x00!.
\item Je kan in een bitsets aliassen definieren voor combinaties van
gezette vlaggen.
\item Gebruik geen bitsets als er ongeldige combinaties van vlaggen
bestaan.
\end{enumerate}

\begin{lstlisting}[float,caption=Bitsets]
[Flags]
public enum Niveaus
{
	Geen = 0x00,
	Plaatselijk = 0x01,
	Gewesteijk = 0x02,
	Verbondelijk = 0x04,
	Nationaal = 0x08,
	Alles = Plaatselijk | Gewestelijk | Verbondelijk | 
		Nationaal,
	Kader = Alles & ~Plaatselijk
}
\end{lstlisting}

\subsection{Membervariabelen}
\index{membervariabelen!declaratie}
\begin{enumerate}[resume]
\item Membervariabelen zijn altijd \lstinline !private!\index{private!membervariabelen}.  Als ze van
buitenaf lees- of schrijfbaar moeten zijn, dan moet dit via property's gebeuren.
\item Gebruik geen membervariabelen voor iets dat je ook in een lokale variabele kwijt kunt.
\item Slechts \'e\'en membervariabeledeclaratie per lijn
\item In een klassedeclaratie zet je de membervariabelen vooraan.\footnote{Op die manier is het direct duidelijk of de klasse stateful of stateless
is.}
\end{enumerate}

\subsection{Lokale variabelen}
\index{lokale variabelen!declaratie}
\begin{enumerate}[resume]
\item Declareer een lokale variabele in de juiste scope, zo dicht mogelijk
bij zijn eerste gebruik.
\item Meerdere lokale variabelen mogen op \'e\'en lijn gedeclareerd worden,
op voorwaarde dat ze niet op die lijn ge"initialiseerd\index{lokale variabelen!initialisatie}
worden.
\end{enumerate}

\subsection{Constanten}
\index{constanten}
Voorbeelden in listing \ref{lblLstConst}.
\begin{enumerate}[resume]
\item Declareer alle constanten behalve  \lstinline !0!, \lstinline !1!,
\lstinline !true!\index{true!vergelijken met}, \lstinline !false!\index{false!vergelijken met} en \lstinline !null!\index{null}. 
\item Gebruik `\lstinline !true!' en `\lstinline !false!' enkel om te
assigneren\index{bool}; vergelijk niet met `\lstinline !true!' of
`\lstinline !false!'.
\item Als een constante een berekende waarde\index{berekende waarde} is, definieer ze dan als die
berekening, en niet als die waarde.
\item Als er een logische connectie\index{constanten!relaties tussen} bestaat tussen constanten, definieer
dan de ene constante in functie van de andere.
\item Evalueer of je het beste gebaat bent met een publieke constante of een public static readonly member. Gebruik `\lstinline !const!' enkel voor `echte' constanten, in de
andere gevallen gebruik je 
`\lstinline !readonly!'\index{constanten!vs readonly}\index{readonly}.\footnote{\lstinline !const!-expressies worden
berekend en gesubstitueerd bij het compileren, \lstinline !readonly!-expressies at runtime.} 
\end{enumerate}
\begin{lstlisting}[float, caption=Constanten, label=lblLstConst]
// Berekende en gerelateerde constanten

// FOUT:

return 86400;

// GOED:

const int SECONDEN_IN_EEN_UUR = 60*60;
const int SECONDEN_IN_EEN_DAG =
  SECONDEN_IN_EEN_UUR * 24;

/* ... */

return SECONDEN_IN_EEN_DAG;


// Niet vergelijken met true of false

// FOUT:
if (mgr.IsGavGroep(GroepID) == true)
{
	// ...
}

// GOED:
if (mgr.IsGavGroep(GroepID))
{
	// ...
}

\end{lstlisting}

\subsection{Strings}
\index{strings}
\begin{enumerate}[resume]
\item Geen letterlijke strings in programmacode.  Declareer ze bij de
resources.\footnote{In het begin van het ontwikkelen hoeft dat niet
direct, maar zodra code wat vastere vorm krijgt wel.}  (Zie listing \ref{lstStrLit}.)
\item Gebruik `\lstinline !String.Format!'\index{String.Format} bij het concateneren van strings
als dit de leesbaarheid vergroot.
\item Gebruik de klasse `\lstinline !StringBuilder!'\index{StringBuilder} als je een groot aantal stringmanipulaties
moet uitvoeren op performantiekritsche plaatsen.
\item Gebruik `\lstinline !String.Compare!'\index{String.Compare} om
strings te vergelijken.\footnote{\lstinline !String.Compare! is
`cultuurgevoelig'.}
\item Gebruik `\lstinline !String.Empty!'\index{String.Empty} in plaats van \lstinline !""!
\item Om te controleren op de lege string, gebruik je \lstinline
!String.IsNullOrEmpty!\index{String.IsNullOrEmpty}.
\addtocounter{enumi}{1} 
% De recommendeation ivm String.Equals valt weg, owv die ivm
% String.Compare 
\item Gebruik het prefix `@' voor een letterlijke string, in plaats van te escapen.
Bijvoorbeeld: \lstinline !@"Backslashes (\) zijn cool."!
\end{enumerate}

\subsection{Arrays en collecties}
\index{arrays}
\index{collecties}
\begin{enumerate}[resume]
\item Indien mogelijk, gebruik de interfaces 
\lstinline !IEnumerable<T>!\index{IEnumerable}
(voor rijen) en \lstinline !IList<T>!\index{IList} (voor lijsten).
\item [165-b.] Gebruik LINQ om collecties te doorzoeken (geen loops).\index{LINQ}\index{collecties!zoeken}
\item Cre\"eer de juiste collectie voor het juiste doel

\begin{tabular}{|c|l|}
\hline
vast aantal elementen met index & array \\
elementen met index & \lstinline !List<T>!\index{List} \\
zoeken mogelijk & \lstinline !Dictionary<TKey, TValue>!\index{Dictionary} \\
gesorteerd & \lstinline !SortedDictionary<TKey, TValue>!\index{SortedDictionary} \\
FIFO & \lstinline !Queue<T>!\index{Queue} \\
LIFO & \lstinline !Stack<T>!\index{Stack} \\
\hline
\end{tabular}
\end{enumerate}


\subsection{Klasses}
\index{klasses}
\index{klasses!declaratie}

\subsubsection{Access modifiers}
\index{klasses!access modifiers}

\begin{enumerate}[resume]
\item Klasses die niet zichtbaar mogen zijn buiten de assembly, krijgen de 
access modifier `\lstinline !internal!'\index{internal}
\item Gebruik de access modifier `\lstinline !public!'\index{public} enkel als het strict noodzakelijk is.
\item Als een type enkel \lstinline !public! moet zijn opdat een bepaalde andere assembly
het zou kunnen gebruiken, kan je `friend assembly's'\index{assembly's!friend}\index{friend assembly's} defini\"eren via het
attribuut \lstinline ![InternalsVisibleTo]!\index{internalsvisibleto@[InternalsVisibleTo]}.
\end{enumerate}

\subsubsection{Statische members}

\begin{enumerate}[resume]
\item Klasses met statische members\index{class members!statische} hebben een statische default constructor.
\item Klasses met enkel statische members krijgen een \lstinline !private!\index{private!constructor} constructor.
\end{enumerate}

\subsubsection{Abstracte klasses}
\index{klasses!abstracte}\index{abstracte klasses}
\begin{enumerate}[resume]
\item Een abstracte klasse heeft enkel \lstinline !protected!\index{protected!constructor} of \lstinline
!private! constructors\index{constructors!abstracte klasses}\index{private!constructor}.
\item Indien mogelijk, voorzie dan een minimale implementatie van een
abstracte klasse.  Deze klasse krijgt het suffix `\lstinline !Base!'.
\end{enumerate}

\subsubsection{Statische en verzegelde klasses}
\index{klasses!statische}\index{klasses!verzegelde}
\begin{enumerate}[resume]
\item Cre\"eer niet te veel statische klasses.  Bekijk eerst of er nieuwe
functionaliteit aan een statische klasse toegevoegd kan worden.
\item Geen \lstinline !protected! members\index{protected!members} voor \lstinline !sealed! klasses.
\item Vermijd \lstinline !sealed! klasses zo veel mogelijk; verzegel liever
members.
\end{enumerate}


\subsubsection{Gebruik van \lstinline !this! en \lstinline !base!}
\index{this}
\index{base}
\begin{enumerate}[resume]
\item Gebruik \lstinline !this! enkel om andere constructors aan te roepen.
\item Als je \lstinline !this! wil gebruiken om name clashes\index{namen!clash} op te lossen,
verander dan \'e\'en van de conflicterende namen.
\item Gebruik \lstinline !base! enkel om een andere constructor of een
`predecessor method' aan te roepen.
\item Je mag enkel een method van \lstinline !base! aanroepen vanuit
dezelfde method in een ervende klasse.\index{methods!predecessor}
\end{enumerate}


\subsection{Methods en property's}
\index{methods}
\index{property's}

\subsubsection{Wanneer methods, wanneer property's?}
\index{methods!vs property's}
\index{property's!vs methods}
\begin{enumerate}[resume]
\item Het lezen van een property mag nooit resulteren in het wijzigen van
een object.
\item Gebruik methods \ldots
\begin{itemize}
\item voor transformaties\index{transformaties} en conviersies\index{conversies}, zoals bijv. `\lstinline !ToXml()!' en `\lstinline !ToSql()!'.
\item als de gevraagde waarde niet gecachet wordt, maar duur is om te berekenen.
\item als de terugkeerwaarde kan verschillen als ze meerdere malen na
elkaar berekend wordt.
\item als het resultaat een nieuwe kopie is van een bestaand object.
\item als een getter overbodig is.  Gebruik geen write-only property's\index{property's!write-only}.
\end{itemize}
\item In alle andere gevallen waar je kan kiezen tussen property's en methods, kies je voor property's.
\end{enumerate}

\subsubsection{Resultaat}
\begin{enumerate}[resume]
\item Als een method een resultaat\index{resultaat}\index{methods!resultaat} teruggeeft, dan moet dit resultaat
semantisch relevant zijn.  De terugkeerwaarde van een method mag nooit een
foutcode\index{foutcodes} zijn;\footnote{Tenzij in een `Try*-pattern'} gebruik hiervoor een `\lstinline !out!'-parameter\index{out}, of een
exception\index{excepties}.
\item Als een method een collectie oplevert, geef dan nooit 
\lstinline !null! als resultaat, maar wel een lege
collectie\index{collecties!lege}.
\end{enumerate}

\subsubsection{Method parameters}
\index{methods!parameters}
\begin{enumerate}[resume]
\item Niet meer dan 5 argumenten voor een method\index{methods!aantal parameters}.  Maak zo nodig een \lstinline !struct!\index{struct} als
er meer informatie doorgegeven moet worden.
\item Niet meer dan twee \lstinline !out!-\index{out} of 
\lstinline !ref!-parameters\index{ref}.  Gebruik zo nodig een \lstinline !struct!.
\item \lstinline !out!- en \lstinline !ref!-parameters komen als laatste
in de parameterlijst
\item Gebruik het minst afgeleide mogelijke type voor method arameters.
\end{enumerate}

\subsubsection{Method overloading}
\index{methods!overloading}
\index{overloading!methods}
\begin{enumerate}[resume]
\item Als methods een gelijkaardige functionaliteit hebben, gebruik dan
overloads.
\begin{lstlisting}
// FOUT:
void DoeHet();
void DoeHetMetItem(Item item);
void DoeHetMetLijst(IEnumerable<Item> items);

// JUIST:
void DoeHet();
void DoeHet(Item item);
void DoeHet(IEnumerable<Item> items);
\end{lstlisting}
\item Standaardiseer zo veel mogelijk de volgorde van de parameters in
overloads en in gelijkaardige methods.\index{consistentie}\index{methods!gelijkaardige}\index{gelijkaardige methods}\index{parameters!volgorde}
\item Overloads met minder parameters roepen overloads met meer
parameters aan, met standaardwaarden voor de ontbrekende parameters.

\end{enumerate}


\subsubsection{Extension methods}
\index{extension methods}
\begin{enumerate}[resume]
\item Vermijd het gebruik van extension methods als je de methods even goed rechtstreeks aan de klasse kan toevoegen.
\item Definieer geen extension methods op \lstinline !object!.
\item Extension methods mogen enkel in dezelfde namespace\index{namespaces!extension methods} als het
ge\"extende type gedefinieerd worden als je methods toevoegt aan een
interface.  In andere gevallen definieer je ze in een andere namespace.
\item Plaats extension methods in namespaces waarvan de naam aangeeft waar
de extension method mee te maken heeft.
\item Gebruik extension methods in deze gevallen
\begin{itemize}
\item Implementatie van methods die relevant zijn voor iedere
implementatie van een bepaalde interface.\index{interfaces!extension methods}
\item Als je voor de implementatie van een method een dependency moet
leggen die niet overeenkomt met het lagenmodel, implementeer die method
dan ergens anders als extension method.
\end{itemize}
\end{enumerate}


\subsubsection{Virtuele methods}

\begin{enumerate}[resume]
\item Maak virtuele methods\index{virtuele methods}\index{methods!virtuele} \lstinline !protected!\index{protected!virtuele methods}.  Als een virtuele
method publiek toegankelijk moet zijn, maak dan een publieke
method die pre- en postcondities test, en de virtuele method aanroept.  (Zie
listing \ref{virtueel}.)  De virtuele helperfunctie krijg het prefix \lstinline !Do!.
\begin{lstlisting}[float,caption=Virtuele methods publiek
toegankelijk maken,label=virtueel]
public T Ophalen(int ID)
{
  Debug.Assert(ID > 0);
  DoOphalen(ID);
  Debug.Assert(T.ID == ID);
}

protected virtual T DoOphalen(int ID)
{
  // Hier gebeurt het echte werk.
}

\end{lstlisting}
\end{enumerate}

\subsubsection{Constructors}
\index{constructors}
\begin{enumerate}[resume]
\item In een constructor mogen enkel property's gezet worden.
\item Gebruik geen virtuele methods\index{constructors!virtuele methods} in een constructor.\footnote{Mogelijk
gebruikt de implementatie van een ervende klasse dan items die nog niet
ge\"initialiseerd zijn door die ervende klasse.}
\end{enumerate}



\subsection{Interfaces}
\index{interfaces}
\begin{enumerate}[resume]
%\item Definieer interfaces als je meer dan \'e\'en hi\"erarchie nodig hebt.
% weggelaten, want niet duidelijk, en spreekt voor zich
\addtocounter{enumi}{1}
\item Gebruik interfaces om een API te defini\"eren\index{API}.
\item Gebruik geen `marker interfaces'\index{interfaces!marking} (interfaces zonder members), gebruik
hiervoor attributen.\index{attributen}
\item Interfaces bevatten hoogstens 10 members\index{aantal members!interfaces}.  Splits een interface
zo nodig op.
\item Implementeer een interface expliciet\index{interfaces!expliciete implementatie} als dat helpt om de API
overzichtelijk te houden.
\item Elke interface moet minstens \'e\'en keer gebruikt worden in
`non-testing code'.  Zo niet mag de interface weg.
\item Vermijd interfaces te wijzigen die gebruikt worden door projecten die je zelf niet onderhoudt.
\end{enumerate}


\begin{lstlisting}[float,caption=Geen letterlijke strings,label=lstStrLit]
// FOUT:
Console.WriteLine(String.Format(
  "Bent u zeker dat u '{1}' wil verwijderen?", Naam));

// GOED:
Console.WriteLine(String.Format(
	Properties.Resources.BevestigVerwijderenPersoon,
	Naam));
\end{lstlisting}


\subsection{Exception handling}
\index{excepties}
\subsubsection{Excepties defini\"eren}
\index{excepties!definitie}
\begin{enumerate}[resume]
\item Definieer geen nieuw exceptiontype\index{excepties!eigen excepties} voor elke mogelijke fout.
\item Definieer een nieuw exceptiontype als je bijkomende property's wil
bewaren.\index{excepties!property's}
\item Eigen exceptions erven van \lstinline !Exception!, en zijn \lstinline ![Serializable]!.\index{serializable@[Serializable]}
\item Eigen exceptions zijn \lstinline !public!\index{public!excepties}.
\item Liever geen complexe exceptionhi\"erarchie\"en.\index{excepties!hi\"erarchie}  Inheritance van
exceptions kan eventueel wel nuttig zijn als een groep van exceptions tegelijk gecatcht moet kunnen
worden.
\item Een exception voorziet vier standaardconstructors\index{excepties!constructors} zoals in listing
\ref{lblExcepConstr}.
\item Zorg ervoor dat er geen excepties kunnen optreden bij de constructie
van een exceptie.
\item Property's van excepties zijn \lstinline !readonly!.\index{excepties!property's}
\end{enumerate}

\begin{lstlisting}[float,caption=Exceptions defini\"eren,label=lblExcepConstr]
public class GeenGavException: Exception, ISerializable
{
	public GeenGavException() : base() { }
	public GeenGavException(string message): base(message){}
	public GeenGavException(string message, 
		Exception inner) : base(message, inner) { }
	public GeenGavException(SerializationInfo info, 
		StreamingContext context) : base(info, context) { }
}
\end{lstlisting}

\subsubsection{Excepties opwerpen}
\index{excepties!opwerpen}
\begin{enumerate}[resume]
\item Als een method zijn taak niet kan uitvoeren, moet een exceptie
opgeworpen worden.
\item Gebruik bij voorkeur standaardexcepties.\index{excepties!standaard-}
\begin{itemize}
\item \lstinline !InvalidOperationException!
\item \lstinline !NotSupportedException!
\item \lstinline !ArgumentNullException!
\item \lstinline !ArgumentException!
\item \lstinline !ArgumentOutOfRangeException!
\item \ldots
\end{itemize}
\item Als je een \lstinline !ArgumentException!\index{ArgumentException} of een afgeleide daarvan
opgooit, zorg er dan voor dat de property \lstinline !ParamName!\index{ParamName} niet
\lstinline !null! is.
\item Gooi nooit expliciet deze types excepties
\begin{itemize}
\item \lstinline !Exception! (gebruik een ervende exceptie)
\item \lstinline !NullRerferenceException!
\item \lstinline !AccessViolationException!
\item \lstinline !InvalidCastException!
\item \lstinline !IndexOutOfRangeException!
\item \lstinline !StackOverflowException!
\item \lstinline !OutOfMemoryException!
\end{itemize}
\end{enumerate}

\subsubsection{Excepties catchen}
\index{excepties!catchen}
\begin{enumerate}[resume]
\item Gebruik steeds een \lstinline !try!\index{try\ldots catch} \ldots \lstinline !catch!-blok
als je externe resources aanspreekt (netwerk, database, bestanden,
\ldots)
\item Probeer \lstinline !try! \ldots \lstinline !catch! in andere
gevallen zo veel mogelijk te vermijden.
\item Als er zich een exceptie voordoet, geef je een verstaanbare boodschap voor de gebruiker\index{foutboodschappen}\index{excepties!foutboodschappen},
en log je de technische details.\index{excepties!loggen}
\item Catch enkel specifieke excepties; op die manier komen onverwachte
excepties boven bij het testen.
\item Een catch van de basisklasse \lstinline !Exception! kan w\'el bij het aanroepen van
third-party code\index{excepties!third-party code}, om problemen in die third party code op te vangen.
\item Om in een \lstinline !catch!-blok de `gecatchte' exceptie opnieuw te `throwen', gebruik je
\lstinline !throw!\index{throw} zonder argumenten.\footnote{Op die manier wordt de volledige stack trace
behouden.}
\item Het \lstinline !catch!blok mag niet leeg zijn.
\item Beperk de grootte van \lstinline !try! \ldots \lstinline !catch! structuren.  Liever
meerdere blokken dan \'e\'en groot blok.\footnote{Op die manier is het duidelijker in welk deel
van de code een exceptie optreedt.}
\end{enumerate}


\subsection{Compileropties}
\index{compiler!opties}
\begin{enumerate}[resume]
\item Gebruik geen \lstinline !#define!\index{define@\#define} in de code; zet compiler variabelen
bij het compileren.
\item Onderdruk geen waarschuwingen van de compiler.\index{compiler!waarschuwingen}
\item Gebruik zo weinig mogelijk 
\lstinline !#if!\index{if@\#if}-structuren in de code.  Groepeer deze structuren in een statische klasse\index{klasses!statische}
\lstinline !CompilerOpties!.  (Zie listing \ref{lblCompilerOpties}.)
\end{enumerate}

\begin{lstlisting}[float,caption=Compiler options,label=lblCompilerOpties]
public static class CompilerOptions
{
	public static bool Debug
	{
		get
		{
#ifdef DEBUG
			return true;
#else
			return false;
#endif
		}
	}
}
\end{lstlisting}

\subsection{Garbage collection}

\begin{enumerate}[resume]
\item Implementeer de interface `\lstinline
!IDisposable!'\index{IDisposable}\index{disposable} als je klasse externe resources (connecties,
files) gebruikt.
\item Gebruik \lstinline !using!\index{using} als je een disposable object
aanmaakt.
\item Als je zelf de method `\lstinline !Dispose!'\index{Dispose} implementeert, zorg er dan voor
dat deze meerdere keren aangeroepen kan worden.
\item Als er een betere naam\index{Dispose!andere naam} dan `\lstinline !Dispose!' is voor een
\lstinline !Dispose!-functie, implementeer `\lstinline !Dispose!' dan
impliciet, en voorzie een method met de betere naam die `\lstinline
!Dispose!' aanroept.
\item Gebruik geen unsafe code.\index{unsafe code}
\item Roep de garbage collector\index{garbage collector} zelf niet expliciet aan.
\item Vermijd zo veel mogelijk het gebruik van destructors\index{destructors}.\footnote{Destructors vertragen
de garbage collector}
\end{enumerate}

\subsection{ASP.NET}
\index{ASP.NET}
\begin{enumerate}[resume]
\item Vermijd het gebruik van sessievariabelen.\index{sessievariabelen}
\item Als je sessievariabelen gebruikt, controleer dan steeds of ze
niet \lstinline !null!\index{null} zijn.
\item Genereer correct XHTML 1.1\index{webstandaarden}\index{HTML}\index{XHTML} (strict).
\item Gebruik style sheets (css) om de look en feel van een pagina te defini\"eren.\index{CSS}
\item Gebruik nooit user input in een client side javascript.\index{Javascript}\opmerking{en plaats ook nooit
niet-gevalideerde userinput in een gegenereerde webpagina.}
\end{enumerate}



\subsection{ADO.NET}\opmerking{Voor het Chirogroepprogramma gebruiken we Entity Framework voor data access}

\subsubsection{Algemeen}

\index{ADO.NET}
\begin{enumerate}[resume]
\item Connecties\index{connecties} zo laat mogelijk openen en zo vroeg mogelijk opnieuw sluiten.\footnote{.NET is
zelf verantwoordelijk voor connection pooling\index{connecties!connection pooling}.}
\item Zorg ervoor dat de connectie altijd gesloten wordt.
\item Bewaar connection strings\index{connecties!connection string} in het configuratiebestand.
\end{enumerate}

\subsubsection{Voor projecten zonder Entity Framework:}\opmerking{Voor nieuwe projecten gebruiken we
steeds entity framework.}
\begin{enumerate}[resume]
\item Gebruik een `Data Provider Factory'\index{data provider factory} om connecties, commando's, \ldots te
genereren.
\item Gebruik de klasse \lstinline !DbParameter!\index{DbParameter} om query's te parameterizeren.\index{query parameters}
\item Bij excepties log je de volgende zaken:
\begin{itemize}
\item Naam van het commando
\item Het commando zelf
\item De parameters
\item De connection string
\end{itemize}
\end{enumerate}

\begin{lstlisting}[float, caption=ADO.NET zonder Entity Framework]
try
{
	// factory is de provider factory.

	DbConnection connectie = factory.CreateConnection();
	connectie.ConnectionString = connStr;

	DbCommand commando = factory.CreateCommand();
	commando.Connection = connectie;
	commando.CommandText = InschrijvingsQuery;
	
	DbParameter parameter = factory.CreateParameter();
	parameter.ParameterName = "@AdNr";
	parameter.Value = adNr;

	commando.Parameters.Add(parameter);
	commando.ExecuteNonQuery();
}
// handel hier exceptions af
finally
{
	commando.Dispose();
	connectie.Dispose();
}
\end{lstlisting}

\subsection{Overig}

\begin{enumerate}[resume]
\item Lock\index{locks} enkel objecten van het type `\lstinline !Object!'
\item Event handlers\index{event handling} zijn kort, en roepen andere methods op.
\item Event handlers worden enkel gebruikt om hun event(s) af te handelen, en worden voor de rest
niet aangeroepen.\footnote{TODO: verder uitwerken event handling}
\item Gebruik liever geen late binding\index{late binding}.
\item In \'e\'en expressie mag de waarde van een variabele niet meerdere
malen aangepast worden door een prefix- of suffixoperator.\index{++}\index{--}  (Bijv. 
\lstinline !items[i++] = ++i! is fout.)
\end{enumerate}

\section{.NET 3}
\index{object initializers}
\subsection{Object initializers}

\begin{enumerate}[resume]
\item Korte initialisatieblokken op \'e\'en lijn
\item Bij object initialisatie in een declaratie, staat staat het
\lstinline !new!-statement\index{new} bij voorkeur op dezelfde lijn als de
declaratie.
\item Als het \lstinline !new!-statement inline gebruikt wordt, en het
niet op \'e\'en lijn kan, komt ook \lstinline !new! op een nieuwe lijn.
\end{enumerate}
\begin{lstlisting}[float,caption = Object initializers]
// object-initialisatie op \'e\'en lijn
DoeIets(new Categorie { Naam = "Kookploeg", Code = "kok" });

// new statement op dezelfde lijn als declaratie
var commVorm = new CommunicatieVorm
	{
		Nummer = 'johan.vervloet@chiro.be',
		Voorkeur = true,
		IsGezinsGebonden = false
	};

// ook new statement op nieuwe lijn
CommunicatieVorm communicatieVorm =
	new CommunicatieVorm
	{
		Nummer = 'johan.vervloet@chiro.be',
		Voorkeur = true,
		IsGezinsGebonden = false
	};
\end{lstlisting}

\subsection{Delegates en lambda-expressies}
\index{delegates}
\index{anonieme methods}\index{methods!anonieme}\index{lambda-expressies}
\begin{enumerate}[resume]
\item Als een delegate maar \'e\'en- of tweemaal gebruikt wordt, en een
vrij eenvoudige syntax heeft, gebruik dan \lstinline !Func<>!\index{Func} en \lstinline !Expression<>!\index{Expression} in plaats van eigen delegatetypes.
\item Gebruik voor delegates dezelfde regels als voor methods.
\item Definieer anonieme methods bij voorkeur als lambda-expressies.
\item Als een lambda-expressie te complex\index{lambda-expressies!complexe} wordt, definieer dan een extra
method, of gebruik delegates.
\item Korte lambda-expressies kunnen inline.
\item Complexere anonieme methods definieer je over meerdere lijnen.
\end{enumerate}

\begin{lstlisting}[float,caption=Delegates en anonieme methods]
private static void DoeIets(Func<int, int> operatie)
{
  Console.WriteLine(operatie(5));
}

public static void Test()
{
  // korte lambda-expressies inline

  DoeIets(arg => arg * 2);
} 
\end{lstlisting}

\subsection{WCF}
\index{WCF}\index{services}
\begin{enumerate}[resume]
\item Services mogen enkel aangesproken worden met correcte
credentials.\index{beveiliging}\index{credentials}
\item Na aanroep van een serviceoperatie is het systeem in een consistente toestand.
\item Services zijn thread safe\index{thread safety} en kunnen gelijktijdig door verschillende
clients aangeroepen worden.
\item Gebruik \lstinline !ServiceHelper!\index{ServiceHelper}\footnote{Dit
zit in ons eigen framework} om services aan te roepen.
\item Gebruik steeds een interface als servicecontract.
\item Een servicecontract biedt 3 tot 20 methods\index{aantal methods!servicecontract} aan.
\item Gebruik het \lstinline ![DataContract]!\index{datacontract@[DataContract]} attribuut voor data
contracts.
\item Gebruik enkel property's of read-only members als \lstinline
!DataMember!.
\item Verwacht en catch excepties\index{excepties!service call} bij een service call.
\end{enumerate}

\subsection{Entity Framework}
\index{entity framework}
\begin{enumerate}[resume]
\item Een objectcontext\index{objectcontext} zo laat mogelijk openen en zo vroeg mogelijk opnieuw sluiten.\footnote{.NET is
zelf verantwoordelijk voor connection pooling\index{connecties!connection pooling}.}
\item Entity's die buiten de context gebruikt moeten worden, moeten
gedetached zijn.  Gebruik hiervoor \lstinline !MergeOption.NoTracking!,
\lstinline !Detach! of \lstinline !DetachObjectGraph!.\footnote{\lstinline !DetachObjectGraph! is code uit ons eigen framework.}
\item Gebruik LINQ\index{LINQ} to entities om te query'en.
\end{enumerate}

\begin{lstlisting}[float, caption=Entity Framework]
using (ChiroGroepEntities db = new ChiroGroepEntities())
{
  db.GelieerdePersoon.MergeOption = MergeOption.NoTracking;
  result = (
    from gp in db.GelieerdePersoon.Include("Persoon")
    where gp.ID == gelieerdePersoonID
    select gp).FirstOrDefault();
}
\end{lstlisting}

\section{Gedrag van de software}
\index{software!gedrag}
\begin{enumerate}[resume]
\item Geef steeds verstaanbare foutmeldingen\index{foutmeldingen} voor de gebruiker.
\item Voer deze checks uit bij het starten van het programma:
\begin{enumerate}
\item Zijn alle files/resources/connecties\index{bestanden!at runtime}\index{resources}\index{connecties} beschikbaar?
\item Bestaat het configuratiebestand\index{configuratiebestand}? Zo nee: maak het aan met standaardwaarden.
\item Is het configuratiebestand correct?
\end{enumerate}
\item Ga er niet vanuit dat het programma op een bepaalde schijf of in een bepaalde directory\index{directory's}
ge\"installeerd is.
\item Gebruik `isolated storage'\index{isolated storage} als je gebruikersspecifieke informatie moet bewaren.
\end{enumerate}

\section{Niet/onvolledig opgenomen in dit document}

\begin{itemize}
\item GUI\index{GUI}'s
\item Event handling\index{event handling}
\item Nested types\index{types!nested}\index{nesting!types}
\item Operator overloading\index{operatoren!overloading}\index{overloading!operators}
\item Ons framework rond entity framework om multi-tier te kunnen werken.
\end{itemize}

\printindex
\end{document}


